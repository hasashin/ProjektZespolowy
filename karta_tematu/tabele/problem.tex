
\def\lendwa{\listlen\data[2]}
\def\lencztery{\listlen\data[4]}

\begin{center}
    \Large Problem badawczy
\end{center}
\begin{table}[h]
    \centering
    \begin{tabular}{|p{5.3cm}|p{9cm}|}
        \hline
        \bf{Numer problemu badawczego} & \@grupa.\data[1,1] \\
        \hline
        \bf{Skrócona nazwa problemu} & \textit{\data[1,2]} \\ 
        \hline
        \multicolumn{2}{|c|}{\bf{\data[1,3]}} \\ \hline
        
        \setcounter{ct}{1}
        \whiledo {\value{ct} < \listlen\data[2]}
        {
            \putrow{2}{\value{ct}}
            \stepcounter{ct}
        }
        \getelem{2}{\lendwa}{1} & \getelem{2}{\lendwa}{2} \\ \hline
        \multicolumn{2}{|c|}{\textbf{\data[3,1]}} \\ \hline
        \setcounter{ct}{1}
        \whiledo {\value{ct} < \listlen\data[4]}
        {
            \putrow{4}{\value{ct}}
            \stepcounter{ct}
        }
        \getelem{4}{\lencztery}{1} & \getelem{4}{\lencztery}{2} \\ \hline
    \end{tabular}
\end{table}
\vspace{2cm}